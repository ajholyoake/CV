\documentclass[a4paper,10pt]{article}
\usepackage{amsmath}
\usepackage{amssymb}
\usepackage{calc}
\usepackage[numbers]{natbib}
\usepackage[dvips]{graphicx}
%\usepackage[active]{srcltx}
\usepackage{pstricks}
\usepackage{psfrag}
\usepackage{hyperref}


\setlength{\textwidth}{7.1in}
\setlength{\topmargin}{-1.2in}
\setlength{\textheight}{11in}
\setlength{\oddsidemargin}{(\paperwidth - \textwidth)/2-1in}
\setlength{\evensidemargin}{(\paperwidth - \textwidth)/2-1in}
%\setlength{\parindent}{0in}
\setlength{\parskip}{\baselineskip}

\pagestyle{empty}

%\address{Alexander Holyoake\\ Trinity College \\ Cambridge CB2 1TQ \\ ajh203@cam.ac.uk}

%\signature{Your signature}

\begin{document}

\begin{flushright}
Alexander Holyoake \\ Trinity College \\ Cambridge CB2 1TQ \\ ajh203@cam.ac.uk
\end{flushright}

\begin{flushleft} 
Mr. Barney Hassell \\ Head of Vehicle Dynamics \\ McLaren Group Ltd. \\ McLaren Technology Centre \\ Chertsey Road \\ Woking, Surrey \\ GU21 4YH
\end{flushleft}

\begin{flushright}
22\textsuperscript{nd} July 2010
\end{flushright}


\noindent Dear Mr. Hassell,

I wish to apply for the post of Vehicle Dynamics Engineer as advertised in Autosport. I am currently in my final year of studying for a Ph.D. in non-Newtonian (Granular) Fluid Dynamics at the University of Cambridge.

The most rewarding aspect of my Ph.D. research has been seeing theoretically computed results match the data obtained from my experiments, experiencing first hand the predictive nature of a combination of Mathematics and engineering, and presenting them to the scientific community.

I have always had an urge to build and understand things physically, which I have now combined with rigorous analytical skills that were developed during my formal education in the Mathematics department at university. This quickly satisfied my need for rigour, but I soon found that I wanted to apply them to physical situations, and gravitated toward applied mathematics. I have found that my Ph.D. balances both of these needs, but could benefit from an injection of competition and cooperation. I feel that working on cutting edge physical problems at a Formula One team, and especially McLaren, will provide an excellent opportunity to combine all of these aspects.

Conducting postgraduate research has required me to approach multifaceted problems in a logical and structured manner, and provided me with experience in communicating complex ideas to both technical and non-technical audiences. My Ph.D. has had a heavy emphasis in designing and performing experiments, as well as using computers extensively to analyse large amounts of data and solve predictive models. I use MATLAB for the majority of my calculations, and take great pleasure in making my (and others') code run as quickly as possible: extensively employing techniques such as vectorisation, and using the C extensions (MEX) framework to write time critical routines. I believe that the experience I have gained in the last few years could be extremely useful in a competitive environment where such a language is used. 
 
Formula One is a sport that has always interested me, moreso since I began my training in fluid dynamics roughly six years ago. I rapidly became as interested (if not more so) in the engineering behind the cars as the championships and the races themselves. As a result last year I finally joined the Formula Student racing team here in Cambridge. I took the job of designing the air intake manifold, which we are hoping to race in our next competition at the beginning of August. I have found working in such a driven environment to be extremely rewarding - in Silverstone this year we went from chassis to (very nearly) finished car in less than three days - something that the judges thought would be nearly impossible when we turned up. Judging by McLaren's impressive turn around in performance last year, I'm sure that this atmosphere will only be bettered, something that excites me incredibly.

In summary, I feel that I have all the necessary skills and ambition for this position, in particular the programming that I feel my Ph.D. has prepared me very well for. I would cherish the opportunity to contribute to an environment packed with driven, intelligent individuals, and to see the results of the team's work and mine being just that bit better than everyone elses.

I am available to interview at any time, apart from 1\textsuperscript{st} to the 8\textsuperscript{th} of August, where I will be competing in Formula Student in Germany, and from 15\textsuperscript{th} to 27\textsuperscript{th} of August, where I will be continuing my language studies in Greece.

\noindent
Yours sincerely,\\ 

Alexander Holyoake   

\end{document}


