\documentclass[a4paper,10pt]{article}
\usepackage{amsmath}
\usepackage{amssymb}
\usepackage{calc}
\usepackage[numbers]{natbib}
\usepackage[dvips]{graphicx}
%\usepackage[active]{srcltx}
\usepackage{pstricks}
\usepackage{psfrag}
\usepackage{hyperref}


\setlength{\textwidth}{7.1in}
\setlength{\topmargin}{-1.2in}
\setlength{\textheight}{11in}
\setlength{\oddsidemargin}{(\paperwidth - \textwidth)/2-1in}
\setlength{\evensidemargin}{(\paperwidth - \textwidth)/2-1in}
%\setlength{\parindent}{0in}
\setlength{\parskip}{\baselineskip}
\newcommand{\sendfirst}{Daniele}
\newcommand{\sendlast}{Casanova}
\newcommand{\sendtitle}{Dr.}
\newcommand{\sendto}{\sendfirst\ \sendlast}
\pagestyle{empty}

%\address{Alexander Holyoake\\ Trinity College \\ Cambridge CB2 1TQ \\ ajh203@cam.ac.uk}

%\signature{Your signature}

\begin{document}

\begin{flushright}
Alexander Holyoake, \\ 8 St. James Road, \\ Shrewsbury SY2 5YH, \\ ajholyoake@gmail.com.
\end{flushright}

\begin{flushleft} 
\sendto, \\ Lotus F1 Team, \\ Whiteways Technical Centre, \\ Enstone, \\ Oxon OX7 4EE.
\end{flushleft}

\begin{flushright}
30\textsuperscript{th} May 2012
\end{flushright}


\noindent Dear \sendtitle\ \sendlast,

I am writing to you regarding a vehicle dynamics position within Lotus F1 team as advertised in an email sent to the University of Cambridge engineering department as part of your graduate scheme. I have recently finished my Ph.D. in non-Newtonian experimental fluid dynamics at the applied mathematics department at Cambridge. I am now seeking to put my knowledge and skills to good use as a Formula 1 engineer.

I have always had an urge to build and understand things physically, but my mathematical aptitude initially led me to pure maths which satisfied my need for rigour, but I soon found that I wanted to apply mathematics to physical situations. As a result I gravitated toward applied mathematics and eventually to experimental mathematics in my Ph.D. During my doctorate the majority of my time was spent designing and performing physical experiments and writing software to collect, filter and analyse large amounts of data as well as solving predictive models numerically. I have found this process very rewarding as I completed the entire process of data collection, analysis, model development myself, and experienced first hand the predictive nature that a combination of mathematics and engineering can give. Conducting postgraduate research has required me to approach multifaceted problems in a logical and structured manner, and provided me with experience in communicating complex ideas to both technical and non-technical audiences.  However with my fanaticism for F1 leading me to get very heavily involved in the university's formula student team I realised that I thrive when working as part of a team to strict deadlines, especially where there is an element of competition involved. It was immensely satisfying seeing research translated into a physical object that was actually used to good effect. I believe that working on cutting edge physical problems at Lotus F1 would encompass the best aspects of both my Ph.D. and formula student experiences.

  I have used MATLAB for the majority of the software produced in my Ph.D. and take great pleasure in making my (and others') code run as quickly as possible: extensively employing techniques such as vectorisation, and using the C extensions (MEX) framework to write time critical routines. I have also developed data acquisition software in C++ and C. I believe that the experience I have gained in this area over the last few years could be extremely useful in a competitive environment where MATLAB in particular is used. I would love the chance to apply this knowledge in a championship winning team that is well known to have a good working environment with great technical innovation. Lotus' performance so far this year along with the new driver in-the-loop simulator clearly shows that this is a great team at which to have an F1 career!

In summary, I feel that I have all the necessary skills and ambition for this position, in particular the programming that I feel my Ph.D. has prepared me very well for. I would cherish the opportunity to contribute to an environment packed with driven, intelligent individuals, and to see the results of the team's work and mine being just that bit better than everyone elses.

I am available to interview at any time apart from the 25\textsuperscript{th} to the 28\textsuperscript{th} of June.

\noindent
Yours sincerely,\\ 

Alexander Holyoake   

\end{document}


